\documentclass{ctexart}
\usepackage{amsmath} % 用于数学公式
\usepackage{amsfonts} % 用于数学字体
\usepackage{amssymb} % 用于数学符号
\usepackage{graphicx} % 用于插入图片 (如果需要)
\usepackage{array} % 用于更复杂的表格
\usepackage{xcolor} % 用于颜色支持 (listings 需要)
\usepackage{listings} % 用于代码块
\usepackage{geometry} % 用于页面边距设置
\usepackage{hyperref} % 用于超链接 (如果需要)
\usepackage{enumitem} % 用于自定义列表
\usepackage{booktabs} % 用于高质量表格线

\geometry{a4paper, margin=1in}

\title{Chapter 06}
\author{谢胡睿 2400014151 数算(B)24-25春}
\date{\today}

\lstset{
    language=C++,
    basicstyle=\ttfamily\small,
    keywordstyle=\color{blue},
    commentstyle=\color{green!50!black},
    stringstyle=\color{red},
    showstringspaces=false,
    breaklines=true,
    frame=single,
    numbers=left,
    numberstyle=\tiny\color{gray}
}

\begin{document}
\maketitle
\begin{center}
\section*{Chapter 06}
\textbf{谢胡睿 2400014151 数算(B)24-25春}
\end{center}

作业(p199\textasciitilde200):
\begin{itemize}
    \item 复习题:2、3、4、5
    \item 算法题:3、5
\end{itemize}

作业(p248\textasciitilde249):
\begin{itemize}
    \item 复习题:1、2、3、7
    \item 算法题:1
\end{itemize}

\section*{复习题2.}
$Ans:$ 总共需要3次,过程如下:\\
首先,字典总共16个元素,且已经有序。\\
使用二分法则先从中间开始,
\begin{enumerate}
    \item $l=0, r=15, mid=7$ \\
    $D[7]=509 < 612$, 则 $l=7+1=8$
    \item $l=8, r=15, mid=11$ \\
    $D[11]=677 > 612$, 则 $r=11-1=10$
    \item $l=8, r=10, mid=9$ \\
    $D[9]=612 == 612, finish!$
\end{enumerate}

\section*{复习题3.}
因为 $\alpha=0.5$, 则 $p=16/0.5=32$,做取余哈希
\[ H(key) = key \pmod{32} \]
\begin{table}[h!]
\centering
\begin{tabular}{@{}llll@{}}
\toprule
元素 & $key \pmod{32}$ & 地址 & 是否碰撞? \\ \midrule
16   & $16 \pmod{32}=16$   & 16   & No        \\ % 16 mod 32 is 16, not 32
87   & $87 \pmod{32} = 23$  & 23   & No        \\
154  & $154 \pmod{32} = 26$ & 26   & No        \\
170  & $170 \pmod{32} = 10$ & 10   & No        \\
275  & $275 \pmod{32} = 19$ & 19   & No        \\
426  & $426 \pmod{32} = 10$ & 10   & Yes (170) \\
503  & $503 \pmod{32} = 23$ & 23   & Yes (087) \\
509  & $509 \pmod{32} = 29$ & 29   & No        \\
512  & $512 \pmod{32} = 0$  & 0    & No        \\
612  & $612 \pmod{32} = 4$  & 4    & No        \\
653  & $653 \pmod{32} = 13$ & 13   & No        \\
677  & $677 \pmod{32} = 5$  & 5    & No        \\
703  & $703 \pmod{32} = 31$ & 31   & No        \\
765  & $765 \pmod{32} = 29$ & 29   & Yes (509) \\
897  & $897 \pmod{32} = 1$  & 1    & No        \\
908  & $908 \pmod{32} = 12$ & 12   & No        \\ \bottomrule
\end{tabular}
\caption{哈希表构造过程}
\end{table}
碰撞3次。

\section*{复习题4.}
用开地址线性探查法:\\
$426: T[11]=426, T[10]=170$ \\
$503: T[24]=503, T[23]=87$ \\
$765: T[30]=765, T[29]=509$ \\
存储状况如下:\\
$T[0] = 512$ \\
$T[1] = 897$ \\
$T[4] = 612$ \\
$T[5] = 677$ \\
$T[10] = 170$ \\
$T[11] = 426$ \\
$T[12] = 908$ \\
$T[13] = 653$ \\
$T[16] = 016$ \\
$T[19] = 275$ \\
$T[23] = 087$ \\
$T[24] = 503$ \\
$T[26] = 154$ \\
$T[29] = 509$ \\
$T[30] = 765$ \\
$T[31] = 703$ \\
其余均为空

\section*{复习题5.}
顺序,二分易于理解,实现简单(二分需要排序 $O(n\log n)$,且是离线的,不能在线处理)\\
哈希法需要找到一个比较好的函数,坏的函数时间复杂度将趋近于 $O(n)$ \\
哈希法要消耗额外的空间 \\
故三种方法各有利弊,需按需完成。

\section*{算法题3.}
顺序存储结构:
\begin{lstlisting}
typedef struct{
    int key;
    int info;
} Node;
#define MAX 1000
typedef struct{
    Node e[MAX];
    int n;
} Seq;

int func(int key, Seq *seq){
    for(int i=0; i < seq->n; i++){
        if(seq->e[i].key == key){
            if(i){ // 如果不是第一个元素
                Node temp;
                temp = seq->e[i-1];
                seq->e[i-1] = seq->e[i];
                seq->e[i] = temp;
                return seq->e[i-1].info; // 返回移动到前一个位置的元素的info
            }
            else { // 如果是第一个元素,无法前移
                return seq->e[i].info;
            }
        }
    }
    return -1; // 未找到
}
\end{lstlisting}

链式存储:
\begin{lstlisting}
typedef struct LNode { // 结构体名称 LNode
    int data;
    struct LNode *next;
} LNode, *LList; // LNode是类型名, LList是指向LNode的指针类型

bool func(LList *lst, int key){ // lst 是指向头指针的指针
    if(*lst == NULL) return false;
    LNode *prepre = NULL; // 指向pre的前一个节点
    LNode *pre = NULL;    // 指向crr的前一个节点
    LNode *crr = *lst;    // 当前节点,从头节点开始

    while(crr != NULL){
        if(crr->data == key){
            if(pre != NULL){ // 如果找到的节点不是头节点
                // 将crr节点移动到pre节点之前
                pre->next = crr->next; // pre的下一个指向crr的下一个,断开pre和crr
                
                if(prepre != NULL){ // 如果pre不是头节点
                    prepre->next = crr; // prepre的下一个指向crr
                }
                else{ // 如果pre是头节点, crr将成为新的头节点
                    *lst = crr;
                }
                crr->next = pre; // crr的下一个指向pre
                return true;
            }
            // 如果找到的节点就是头节点,它已经是第一个了,无需移动
            return true; 
        }
        prepre = pre;
        pre = crr;
        crr = crr->next;
    }
    return false; // 未找到key
}
\end{lstlisting}

\section*{算法题5.}
\begin{lstlisting}
enum SlotStatus { EMPTY, OCCUPIED, DELETED };

struct HT {
    int key;
    SlotStatus status;
};

#define MAX_TABLE_SIZE 1000 // 假设哈希表最大大小
HT table[MAX_TABLE_SIZE]; // 声明哈希表数组

// 假设 h1 和 h2 是已定义的哈希函数
// 假设 m 是哈希表的大小 (模数)
// init every entry as EMPTY (需要在某个初始化函数中完成)

bool Delete(int key_to_delete) {
    int i = 0;
    int initial_hash_val;
    int step_hash_val;
    int current_address;
    int m = MAX_TABLE_SIZE; // 假设 m 是表的大小

    // initial_hash_val = h1(key_to_delete) % m; // 假设h1已定义
    // step_hash_val = h2(key_to_delete) % m;    // 假设h2已定义
    // 为了编译,这里用简单的示例代替
    initial_hash_val = key_to_delete % m; 
    step_hash_val = 1 + (key_to_delete % (m-1)); // 确保与m互素且不为0

    if (step_hash_val == 0) { // 确保步长不为0 (虽然上面的h2示例已处理)
        step_hash_val = 1;
    }

    do {
        current_address = (initial_hash_val + i * step_hash_val) % m;
        if (table[current_address].status == OCCUPIED && table[current_address].key == key_to_delete) {
            table[current_address].status = DELETED;
            return true;
        }
        if (table[current_address].status == EMPTY) { // 如果遇到空槽,说明元素不存在
            return false;
        }
        i++;
    } while (i < m); // 避免无限循环,最多探测m次

    return false; // 探测完整个表也未找到
}
\end{lstlisting}

\end{document}